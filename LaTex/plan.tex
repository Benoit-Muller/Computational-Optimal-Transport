\documentclass[a4paper]{article}
\usepackage{amsmath} %pour tous les trucs de math
\usepackage{systeme} %pour tous les systemes d'équations
%\usepackage{ bbold }  %pour toutes les doubles lettres
\usepackage{amssymb}  %pour les double Lettres
\usepackage{IEEEtrantools} %pour les équations en collonnes
\usepackage{amsthm} %pour les preuves
\usepackage[english]{babel} % la langue utilisée
\usepackage[utf8]{inputenc} % encodage symboles entrée
\usepackage[T1]{fontenc} % encodage symboles sortie
\usepackage{fancyhdr} %pour les entêtes et pied de page
%\usepackage[math]{blindtext} % pour le Lorem ipsum
%\usepackage{enumitem} %pour changer les listes
\usepackage[a4paper,textwidth=16cm]{geometry}
%\usepackage[framed,numbered]{mcode} %MatLab
\usepackage{graphicx} % pour les graphiques
%\usepackage{subfig} % pour les doubles figures
\usepackage{float} % pour bien positionner les figures
\usepackage[dvipsnames]{xcolor} % pour la couleur du texte et de la page
%\usepackage{biblatex} % bibliographie
%\usepackage{csquotes} % pour que la biblio s'adapte à la langue
\usepackage{prettyref}
\usepackage[hidelinks]{hyperref} % pour les hyperliens et références(mettre en dernier)

\newrefformat{fig}{Figure~[\ref{#1}]}
\newrefformat{it}{question~\ref{#1}.}
\newrefformat{eq}{(\ref{#1})}
\newrefformat{seq}{Section~\ref{#1}}
\newrefformat{th}{Theorem~\ref{#1}}
\newrefformat{lem}{Lemma~\ref{#1}}
\newrefformat{cor}{Corollary~\ref{#1}}
\newrefformat{rem}{Remark~\ref{#1}}

\newtheorem{theoreme}{Theorem} %[section]
\newtheorem{corollaire}{Corollary} %[theorem]
\newtheorem{lemme}{Lemma} %[theorem]
\theoremstyle{definition}
    \newtheorem{definition}{Definition} %[section]
\theoremstyle{remark}
     \newtheorem*{remarque}{Remark}

\pagestyle{fancy}
%\fancyhf{}
\lhead{Computational Optimal Transport: A Comparison of Three Algorithms}
\chead{}
\rhead{Benoît MÜLLER}

\title{ \Huge Computational Optimal Transport:\\ A Comparison of Three Algorithms}
\author{Benoît MÜLLER}
\date{lastly compiled \today}

\DeclareMathOperator*{\argmin}{argmin}
\DeclareMathOperator*{\argmax}{argmax}
\newcommand{\R}{\mathbb{R}}
\newcommand{\N}{\mathbb{N}}
\newcommand{\Z}{\mathbb{Z}}
\newcommand{\C}{\mathbb{C}}
\newcommand{\com}[1]{\textcolor{ForestGreen}{[~\emph{#1}~]}}
%\newcommand{\nom}[nombre d’arguments]{définition(#1,#2,...,#n)}

\begin{document}
\maketitle
%\tableofcontents
\begin{center}
    \Huge \bf{Plan}
\end{center}
\begin{itemize}
    \item Abstract
    \item Introduction
    \begin{itemize}
        \item contributions
        \item The problem: Push-forward, transport map, optimal transport.
        \item Theoretical background: Brenier's theorem, Wasserstein distance, well posedness of the problem in the sense of Hadamar).
    \end{itemize}
\end{itemize}
\section{Linear Programming formulation}
The Kantorovich problem is an infinite dimensional linear problem.
\begin{itemize}
    \item Discretization of measure leads to linear programming
    \begin{itemize}
        \item Transport maps are bijections on the supports.
        \item Such transport leads to an integer linear program on the set of permutations matrices.
        \item We can relax the integer condition to work on the polytope of doubly stochastic matrices .
        \begin{itemize}
            \item vertices of polytope always contain solutions (linear programming theory)
            \item vertices of the polytope of doubly stochastic polytope are permutations matrices (Birkhoff's Theorem)
        \end{itemize}
    \end{itemize}
    This linear program is actually the discretization of the Kantorovich formulation. The integer relaxation is then the discrete version of Brenier's equivalence between Monge and Kantorovich formulation.
    \item Duality and KKT conditions
    Write primal vs dual formulation, strong duality, and KKT conditions
    \item The Hungarian algorithm
    \begin{itemize}
        \item Based on KKT conditions
        \begin{itemize}
            \item dual feasibility and complementary slackness always holds
            \item primal feasibility partially holds
        \end{itemize}
        \item Steps
        \begin{itemize}
            \item Initialize primal and dual variables
            \item Where dual constrain is active, try to complete primal variable. Is equivalent to perfect matching on a subgraph, and is done by finding augmenting trees.
            \item When it fails, update dual variable in a wise way so that it can continue  
        \end{itemize}
        \item Pseudocode: comment and explain.
        \item Complexity: $\mathcal{O}(n^3)$, justify with algorithm loops
    \end{itemize}
    \item Choice of discretization
    \begin{itemize}
        \item Use $x_i\backsim^{i.i.d.}\mu$ and  $y_j\backsim^{i.i.d.}\nu$
        \item convergence/quality of the approximation in the Wasserstein metric.?
        \item Quasi random sampling?
    \end{itemize}
    \item Possible improvements
    \begin{itemize}
        \item What hungarian algorithm doesn't use: the cost matrix comes from the euclidian distance.
        \item Order of the choices of indices implentation: deep-first, depth-first, geospatial order (Hilbert curve and order) 
    \end{itemize}
\end{itemize}

\section{Monge-Ampère equation formulation}
\section{Fluid dynamics formulation}
\end{document}